\section{\name Discussion} \label{sec:dis}

As talked before, when to free pages in the cache pool relies on both the absolute memory size of the cache and the proportion between the cache size and page-table size. Default thresholds of the two factors are determined by a specific setting, where both thresholds are measured in the busy setting (created by a specific stress tool) when no pages are freed to the buddy system. Therefore, measurements of the thresholds are heavily dependent on a specific setting, which may not work in other situations. If the thresholds are not set appropriately, freeing pages will occur often, thus causing IOTLB flush. This is why we provide an interface for users to manually modify the default values of both thresholds. Nevertheless, the approach of developing the interface is not flexible enough to control the cache size and we plan to put forward a self-adaption algorithm in the future work. Basically, the algorithm is invoked periodically, adjusting the memory usage of the cache pool according to both the memory page number of created page tables of a target application and the frequency of IOTLB-flush. When users dynamically enable the cache mechanism for the application, the algorithm could automatically initialise the absolute memory usage by scanning the page numbers in cache and in page tables and also determine the proportion between them. Note that the proportion differs in each level of cache, since PTE is used more often than both PGD and PMD. If IOTLB is found out to be flushing during the running period, the algorithm will appropriately increase the threshold of the total memory size. If the cache is beyond a specific percentage of the application's whole memory size, the algorithm is supposed to free the exceeded memory pages even if the frequency of IOTLB flush does not drop to zero.

%has the properties as follows.
%\begin{enumerate}
%\item (P1) Frequency of IOTLB flush will reach the zero level as soon as possible.
%\item (P2) Memory usage of the cache pool is under control.
%\item (P3)
%\item (P4)
%\end{enumerate}

%\zhi{Macrobenchmark results of netperf do not reveal IOTLB's impacts on the DMA transactions.}
