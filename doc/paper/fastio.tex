% TEMPLATE for Usenix papers, specifically to meet requirements of
%  USENIX '05
% originally a template for producing IEEE-format articles using LaTeX.
%   written by Matthew Ward, CS Department, Worcester Polytechnic Institute.
% adapted by David Beazley for his excellent SWIG paper in Proceedings,
%   Tcl 96
% turned into a smartass generic template by De Clarke, with thanks to
%   both the above pioneers
% use at your own risk.  Complaints to /dev/null.
% make it two column with no page numbering, default is 10 point

% Munged by Fred Douglis <douglis@research.att.com> 10/97 to separate
% the .sty file from the LaTeX source template, so that people can
% more easily include the .sty file into an existing document.  Also
% changed to more closely follow the style guidelines as represented
% by the Word sample file.

% Note that since 2010, USENIX does not require endnotes. If you want
% foot of page notes, don't include the endnotes package in the
% usepackage command, below.

% This version uses the latex2e styles, not the very ancient 2.09 stuff.
\documentclass[letterpaper,twocolumn,10pt]{article}
\usepackage{usenix,epsfig,endnotes,array}
%\usepackage{subfigure}
%\usepackage{subfig}
\usepackage{subcaption}
\usepackage{graphicx}
%\usepackage[demo]{graphicx} % omit 'demo' for real document
\usepackage[normalem]{ulem}
\usepackage{multirow}    % Use to create table with a column that spans multiple rows
\usepackage{pifont}      % Provides the ding symbol used for comments
\usepackage{color}       % Used to highlight comments
\usepackage{xspace}      % Intelligently adds space after a word via \xspace
\usepackage{flushend}    % balance the last page

\begin{document}

\newcommand{\name}{PTCache\xspace}
\newcommand{\eat}[1]{}  %% for quick commenting of a large trunk of texts
\newcommand{\authcomment}[3]{\textcolor{#3}{#1 says: #2}}\newcommand{\yueqiang}[1]{\authcomment{Yueqiang}{#1}{red}}
\newcommand{\zhi}[1]{\authcomment{Zhi}{#1}{red}}



%don't want date printed
\date{}

%make title bold and 14 pt font (Latex default is non-bold, 16 pt)
%\title{\Large \bf Improving I/O Performance of All Peripheral Devices on Paravirtualized Platforms}
\title{\Large \bf PTCache: Page Table Management Improvements For Performance Benefits }

%for single author (just remove % characters)
\author{
{\rm Your N.\ Here}\\
Your Institution
\and
{\rm Second Name}\\
Second Institution
% copy the following lines to add more authors
% \and
% {\rm Name}\\
%Name Institution
} % end author

\maketitle

% Use the following at camera-ready time to suppress page numbers.
% Comment it out when you first submit the paper for review.
\thispagestyle{empty}


\subsection*{Abstract}
%As a key virtualization technique, the paravirtualization technique is widely applied to many existing cloud platforms.
When the mainstream operating systems, such as Linux, move onto the paravirtualized platforms, their page table management components are required to be properly patched.
However, the existing updates mainly focus on the security enhancements, without enough attentions for the performance improvements. 
Based on our observations, there are two main performance issues. 
The first one is the additional dependence between the page table (de)allocations and the IOTLB flushes, which inevitably introduces extra IOTLB misses, and would consequently have negatively impacts on the I/O performance of all peripheral devices.
The other one is the long execution paths for page table allocations and deallocations, which directly leads to the operating time of the page table creations and destructions, and consequently increases the latency of the creations and exits of processes.


In this paper, we propose Page-Table Cache (\name), a novel software-only approach for improving the performance in page table management.
First, \name eliminates the additional IOTLB misses with a fine-grained validation scheme, which separates the DMA and software validations, instead of doing them all in one.
Second, \name reduces the length of the execution path of the page table (de)allocations by a page table cache, which maintains a dedicated buffer for serving page table allocations and deallocations. 
We implement a prototype on Xen with Linux as the guest VM's kernel. We do small modifications of Xen version 4.2.1 (xxx SLoC) and Linux kernel version 3.2.0 (xxx SLoC). 
We evaluate the I/O performance in both micro and macro ways. 
The micro experiment results indicate that \name is able to effectively eliminates the additional IOTLB misses, from xxx to zero, and reduces the page table (de)allocations, (xx\%, xx\%), (xx\%, xx\%) and (xx\%, xx\%) improvements for allocation and deallocation pairs for three-levels page table, from top to bottom.
The macro benchmarks shows that the latencies of the processes creation and exit are reduced as expected, from xx\% to xx\% on average.


\section{Introduction} \label{sec:intro}
In the paravirtualization~\cite{XEN-SOSP03,denali-paravirtualization}, the kernel of each guest Virtual Machine (VM) and the hypervisor share the same virtual space including many security-critical data structures, like page tables and Global Descriptor Table (GDT). All these data structures are managed by the guest VM, but their updates are intercepted and validated by the hypervisor~\cite{XEN-SOSP03} with the purpose of preventing malicious accesses from the guest software.
However, the paravirtualization technology itself is not armed with efficient protection to prevent DMA attacks~\cite{disaggregation}.
To fix this gap, the hypervisor has to resort to the I/O virtualization (AMD-Vi~\cite{amdvt} or Intel VT-d~\cite{intelvt}) technology, which introduces a new Input/Output Memory Management Unit (IOMMU) to restrict DMA accesses on the physical memory addresses occupied by the hypervisor and the shared security-critical data structures.
Leveraging the combination of the paravirtualization and I/O virtualization, the hypervisor prevents all malicious accesses from processor and hardware peripheral devices.

%Highlight the problem, and emphasise the importance.
\textbf{Problem.} Although the combination of the paravirtualization and I/O virtualization brings strong security protection, we surprisingly find out that it slows down the speed of DMA address translation due to the additional I/O translation look-aside buffer (IOTLB) flushes.
Specifically, the guest page tables should be protected from every DMA access.
Thus, when a writable page\footnote{A \emph{writable page} is the page in the guest VM, allowing the guest VM to freely read and write according to its own purpose.} turns into a page-table page, the hypervisor has to update the IOMMU page table and flush an IOTLB entry to prevent DMA from accessing it.
The IOTLB flush is necessary for the sake of the security of the hypervisor, but it inevitably increases the miss rate of IOTLB, and consequently reduces I/O performance, especially for the high-speed devices.
Similarly, when a page-table page is turned to a writable page, the hypervisor needs to return the ownership to the guest VM.
As a result, the hypervisor has to update the IOMMU page table and flush IOTLBs for allowing the guest VM to freely access the returned writable page.
In addition, these page-type changes between page-table pages and writable pages are \emph{often} triggered during the whole life cycle of a running system.
As a consequence, the IOTLB flushing events are \emph{frequently} triggered, which inevitably lower the speed of the DMA address translation and further badly affect I/O performance of all peripheral devices.

%existing work

%introduce our approach
This new dependency issue between guest page tables and the DMA transferring urges us to reshape the design of the hypervisor and the guest operating system to minimize the effects on the I/O performance.
In response to this problem, we aim to improve the guest kernel and the hypervisor to reduce the flush rate of IOTLB, and therefore ameliorate its impacts on the I/O performance of all peripheral devices.

Our approach rests on the observation that the high IOTLB flush rates experienced by a guest VM
are due to the page-type changes generated by the numerous creations and destructions of page tables.
We cannot reduce the number of the creation/destruction of page table, but reducing the number of page-type change is possible.
Inspired by this observation, we propose the page-table cache, which queues the released/freed page-table page in the hope
it will be reused (popped out the cache) in the creation of page table in the near future.
During this process, the destruction and the creation of the page table will not introduce page type change, and therefore there will be no IOTLB flush.
Besides minimizing the IOTLB flush rate, the page-table cache also accelerates the creation and destruction of page table.
The reason is that the kernel does not need to ask for pages from enormous and costly memory management subsystem, instead it could quickly get pages from the page-table cache with very small cost.
In order to allow the kernel memory management subsystem to use the pages cached in the page-table cache, we add an interface for the page-table cache to free pages.

We implement the page-table cache with small modifications of hypervisor Xen version 4.2.1 (xxx SLoC) and Linux kernel version 3.2.0 (xxx SLoC), and evaluate the I/O performance in micro and macro ways.
The micro experiment results indicate that the new page-table cache is able to effectively reduce the miss rate of IOTLB (from xxx to zero) with \emph{less CPU usage}, even when the page tables are frequently updated.
The macro benchmarks shows that the I/O devices always produce better (or the same) performance, even when the system frequently generates many temporal processes.


In particular, we make the following contributions:
\begin{enumerate}
\item We are the first, to the best of our knowledge, to identify the dependency issue between guest page table and DMA transferring.
\item We proposed a novel approach - page table cache, to improve I/O performance of all peripheral devices by minimizing the flush rate of IOTLB, without sacrificing the system security.
\item We implemented a prototype of the page table cache and evaluated the performance in both micro and macro ways.
\end{enumerate}

The rest of the paper is structured as follows: In Section~\ref{sec:preli} and Section~\ref{sec:prob}, we briefly describe the background knowledge, and highlight our goal and the thread model. In Section~\ref{sec:rationale} we discuss the design rationale. Then we describe the system overview and implementation in Section~\ref{sec:overview} and Section~\ref{sec:implement}. In Section~\ref{sec:eva}, we evaluate the security and performance of the system, and discuss several attacks and possible extension in Section~\ref{sec:dis}. At last, we discuss the related work in Section~\ref{sec:related}, and conclude the whole paper in Section~\ref{sec:con}.


\section{Problem Definition} \label{sec:prob}
In this section, we describe the identified performance issues: 1) long execution paths of the guest page table (de)allocation and 2) the additional IOTLB flushes.
As Xen~\cite{XEN-SOSP03} is a typical and popular paravirtual hypervisor, we use Xen in a x86 MMU model~\cite{x86-pv-model} to illustrate the details of issues. 
%It is easy to map the descriptions into the corresponding mechanisms on other paravirtual hypervisors.

\subsection{Long Execution Path}\label{sec:longpath}
The long execution path issue refers to the execution paths for installing/uninstalling a the guest page table page.
In the current design, installing/uninstalling a page table page has to invoke the complex memory allocators and performance both software (i.e., page table) and DMA validations.
%In addition, the DMA validations always lead to additional IOTLB flushes, which would reduce the DMA address translation speed and may consequently lower the I/O performances.
In addition, such installation and uninstallation events are frequently happened in the system, e.g., the process creation and exits will result in many page-table page installations/uninstallations.
Thus, it is necessary to shorten the execution paths to save CPU usage, benefiting 

%As the uninstallation process of the guest page table page is the reverse to the installation process. In this section, we only present the details of installation.
 
\subsubsection{Costly Memory Allocators}
To allocate a page-table page, the guest kernel has to invoke the system allocators, typically from slab allocator to buddy allocator.
There are many functions are 
Each of these allocators maintains a set of complex data structures to track the internal status.
%not only returns or release the page, the internal status will be updated correspondingly.
any allocation and deallocation invocation will trigger the updates of such data structures. 
% in addition, to support multicore setting, such allocators have to leverage locks to prevent race conditions.
% in one invocation, there will be many locks locked and unlocked 

All these increase the invocation time.

Similarly, to free a page-table page, the guest kernel also has to leverage the system allocators to return the page back to the system.


\subsubsection{Security Validations}
Once getting a page, it is not ready to install.

\paragraph{Page Table Validations}\label{sec:pv-security}

Page tables are used by a hardware, i.e., Memory Management Unit (MMU), to translate the linear addresses into physical addresses used by the hardware to execute instructions.
In the PAE-enabled paging mode, a page table has three levels: L1 level (bottom level), L2 level (middle level) and L3 level (top level).
The slots in L1, L2 and L3 levels are known as Page Table Entry (PTE), Page Middle Directory (PMD) and Page Global Directory (PGD), respectively.
A PTE slot could determine the access permissions of a page, e.g., the kernel could set a page as read-only by clearing the bit within a PTE slot that represents the writable permission.

There are many page tables in a guest VM, as each user process has its own page tables.
The creation and exit of a user process will be accompanied by the creation and destruction of a page table respectively.
It means that if there are numerous temporary processes generated within a period, there will be a large number of page tables created.

In order to ensure that the guest cannot subvert the system, Xen requires that certain update policies are maintained,
and thus all updates of the page tables should be vetted by Xen.
To this end, the guest OS is deprivileged, from ring-0 to ring-1, leaving ring-0 for the Xen hypervisor.
This prevents the guest OS from executing privileged instructions, e.g., the guest OS cannot directly update control registers.

Xen also defines a number of page types, which are listed in Table~\ref{tab:pagetype}, and maintain a type reference count for each page.
Xen enforces the policies that any given page has exactly one type at any given time,
and only pages with the writable type have a writable mapping in the page tables.
By doing this it can ensure that the guest OS is not able to directly modify any page-table pages and therefore subvert the security of the whole system.
If the guest kernel attempts to update the page table, it has to issue a hypercall to ask the hypervisor to complete the update.
As Xen is always involved in all updates of the page tables, the policies on the page table updates are non-bypassable.

Whenever a page table is loaded into the hardware page-table base register (cr3),
the hypervisor takes an appropriate type reference with the L3 page-table type.
If the page is not already of the required type, then in order to take the initial reference it must first have a type count of zero.
In addition, it must be validated to ensure that it follows the following policy:
for a page with a page-table type to be valid, it is required that any pages referenced
by a present page table entry in the page have the type of the next level down.
For instance, any page referenced by a page with type L3 Page Table must itself have the type L2 Page Table.
This policy is applied recursively down to the L1 page table layer.
At L1 the invariant is that any data page mapped by a writable page table entry must have the writable page type.
By applying these policies, Xen ensures that all page-table pages as a whole are safe to be loaded into the cr3.
Similar requirements are also placed on other special page types, e.g., GTD/LDT pages.

The page type is allowed to be changed.
Xen enforces that the type of a page can only be changed when the type count is zero.
In addition, Xen also requires that every page type update only occur between writable and non-writable pages, as summarized in Figure~\ref{fig:page-type-updates}.

\begin{figure}[ht]
\centering
\includegraphics[width=0.45\textwidth]{image/background/page-type-updates.png} \\
\caption{Page Type Updates between writable pages and non-writable pages. (Writable pages are writable for both software and DMA
while non-writable pages are inaccessible to them.)}
\label{fig:page-type-updates}
\end{figure}


\paragraph{DMA Validations}

\subsection{Additional IOTLB Flush Issue}
The dependency between the guest page table and the IOTLB is very subtle.
To fully understand the dependency, we need to know the details about how the paravirtualized hypervisor protects itself through guest page table and IOMMU.
Specifically, we explicitly describe 1) paravirtualized MMU mode, and 2) IOMMU and DMA address translation.

\subsubsection{DMA Address Translation}
%citing: intel vt-d
The input/output memory management unit (IOMMU)~\cite{directio} is a memory management unit (MMU) that connects a DMA-capable I/O bus to the main memory.
Like a traditional MMU, the IOMMU maps device addresses (also called as I/O addresses) to physical addresses through a dedicated page table.
This technique is also known as DMA remapping.
The IOMMU page table that is created and maintained by the hypervisor in its own space, is able to restrict the access on a particular page by configuring the permission bits.
The hypervisor grants different access permissions for different page types, such as the writable pages are always allowed full access permissions, while the page-table pages are always inaccessible to any devices.
This is why when the page-type changes between writable page and page-table page, updating IOMMU page table is always necessary.

However, the DMA remapping always needs the page table walking, which is slow and inefficient.
To accelerate the translation speed, the I/O translation look-aside buffer (IOTLB) is introduced.
The IOTLB is used to cache frequently accessed page table entries.
By doing so, the IOTLB is very likely to be accessed, indicating that the physical address of a queried DMA address will be immediately fetched through the IOTLB path (Figure~\ref{fig:subfig:a}).
If unlikely the IOTLB miss occurs, the DMA remapping still can go the slow page-table path to get the physical address (Figure~\ref{fig:subfig:b}).
To achieve a better I/O performance, the DMA remapping should avoid taking the page-table path as far as possible.

\subsubsection{Negative Impacts}

\begin{figure}[!t]
    \begin{subfigure}{0.5\textwidth}
        \includegraphics[width=1\textwidth]{image/background/DMA-IOTLB-translation.png}
        \caption{\centering IOTLB Path.}
        \label{fig:subfig:a}
    \end{subfigure}%
    \vfill \vfill \vfill \vfill
    \begin{subfigure}{0.5\textwidth}
        \includegraphics[width=1\textwidth]{image/background/DMA-pt-translation.png}
        \caption{\centering I/O Page-Table Path.}
        \label{fig:subfig:b}
    \end{subfigure}
    \caption{IOTLB path is orders of magnitudes faster than I/O Page-Table Path.}
    \label{fig:dma-add-trans}
\end{figure}

%talking about how to flush IOTLB
%invalidation requests and invalidation interface
According to IOMMU specification~\cite{intelvt}, a typical IOMMU is able to provide three types of IOTLB invalidation schemes, i.e., global invalidation, domain-selective invalidation, page-selective invalidation, which differ in granularity.
Specifically, the global invalidation will always invalidate all IOTLB entries as a whole.
The domain-selective invalidation only invalidates the selected VM domain's IOTLBs, whose performance is a little better than the previous global invalidation.
The page-selective invalidation that only invalidate the corresponding IOTLB could achieve the best performance, comparing with the previous two schemes.
%Intuitively, when a requested entry that corresponds to a specified DMA address needs to be invalidated, a page-selective invalidation is the best choice for the sake of performance.
Besides the invalidation scheme, IOMMU also supports two kinds of invalidation interfaces: register based invalidation and queued invalidation interface, between which queued invalidation performs better.
Whatever granularity IOTLB flushes in, it will inevitably increase the probability of IOTLB misses, thereby introducing negative effects on the I/O performance.
%And this is where our motivation lies.

\section{Motivation} \label{sec:moti}
As we know there are seven page types.
Among of them, segment descriptor pages are rarely updated, as they are almost const structures.  
However, the page-table pages are frequently changed from/to writable pages 
and these changes are driven by the process creations and exits.

Base on the above descriptions, we know that the page-type changes could trigger the invalidation of IOTLB entries.


From the above descriptions, we can summarize the dependency between the guest page table and the IOTLB flushing into two key points. We list them as follows:
\begin{enumerate}
\item (O1) Each page-type change (like the changes in Figure~\ref{fig:page-type-update}) will trigger the invalidation of at least one IOTLB entry.
\item (O2) The main source of causing IOTLB flushing is the page-type changes between writable pages and page-table pages.
\end{enumerate}

\begin{figure}[ht]
\centering
\includegraphics[width=0.5\textwidth]{image/background/wr2pt.png} \\
\caption{The page type updates between writable and page-table pages.}
\label{fig:wr2pt}
\end{figure}




Based on the observations, we are motivated to propose \name in order to reduce as many IOTLB flushes as possible for a maximum possible use of the IOTLB-path while retain the safety.

\section{Problem Definition} \label{sec:prob}
In this section, we first describe the related background, i.e., page allocation and deallocation in guest OS, page table security validations, and the DMA security validations.
Then, we highlight the identified two performance issues: 1) long execution paths of the guest page table (de)allocation and 2) the additional IOTLB flushes.
As Xen~\cite{XEN-SOSP03} is a typical and popular paravirtual hypervisor, we use Xen in a x86 MMU model~\cite{x86-pv-model} to illustrate the details. 
The presented or similar mechanisms would be available on other paravirtual setting.
 
\subsection{Page Allocation and Deallocation}
To allocate a page, the guest kernel has to invoke the system allocators, typically from slab allocator to buddy allocator.
The slab allocator consists of a variable number of caches that are linked together on a doubly linked circular list. 
Each cache maintains blocks of contiguous pages in memory called slabs, which are carved up into small chunks for the data structures and objects. 
When \emph{kmalloc} is called, all it does is searching through the prepared caches. 
If there is no suitable object, the buddy allocator will be involved.
The Buddy allocator manages all free pages that are allocated in blocks with the sizes of powers of 2.  
When the allocation function is invoked, it first searches for blocks of pages of the size requested following the chain.
If no blocks of the requested size are free, blocks of the next size (which is twice that of the size requested) are looked for. 
This process continues until all of the free area has been searched or until a block of pages has been found. 

In contrast to the page allocation, the page deallocation is to return the page back to the system.
The deallocation process may also invoke slab allocator and/or buddy allocator, and would trigger the updates of the corresponding data structures, 
e.g., the buddy allocator always attempts to recombine the freed pages into larger blocks.

In brief, the page allocation and deallocation are time costly due to the deep invocations and complex updates of the dependent data structures.


\subsection{Page Table Validations}\label{sec:pv-security}

Page tables are used by a hardware, i.e., Memory Management Unit (MMU), to translate the linear addresses into physical addresses used by the hardware to execute instructions.
In the PAE-enabled paging mode, a page table has three levels: L1 level (bottom level), L2 level (middle level) and L3 level (top level).
The slots in L1, L2 and L3 levels are known as Page Table Entry (PTE), Page Middle Directory (PMD) and Page Global Directory (PGD), respectively.
A PTE slot could determine the access permissions of a page, e.g., the kernel could set a page as read-only by clearing the bit within a PTE slot that represents the writable permission.
There are many page tables in a guest VM, as each user process has its own page tables.
The creation and exit of a user process will be accompanied by the creation and destruction of a page table respectively.
%It means that if there are numerous temporary processes generated within a period, there will be a large number of page tables created.

In order to ensure that the guest cannot subvert the system, Xen requires that certain update policies are maintained,
and thus all updates of the page tables should be vetted by Xen.
To this end, the guest OS is deprivileged, from ring-0 to ring-1, leaving ring-0 for the Xen hypervisor.
This prevents the guest OS from executing privileged instructions, e.g., the guest OS cannot directly update control registers.

Xen also defines a number of page types, which are listed in Table~\ref{tab:pagetype}, and maintain a type reference count for each page.
Xen enforces the policies that any given page has exactly one type at any given time,
and only pages with the writable type have a writable mapping in the page tables.
By doing this it can ensure that the guest OS is not able to directly modify any page-table pages and therefore subvert the security of the whole system.
If the guest kernel attempts to update the page table, it has to issue a hypercall to ask the hypervisor to complete the update.
As Xen is always involved in all updates of the page tables, the policies on the page table updates are non-bypassable.

Whenever a page table is loaded into the hardware page-table base register (cr3),
the hypervisor takes an appropriate type reference with the L3 page-table type.
If the page is not already of the required type, then in order to take the initial reference it must first have a type count of zero.
In addition, it must be validated to ensure that it follows the following policy:
for a page with a page-table type to be valid, it is required that any pages referenced
by a present page table entry in the page have the type of the next level down.
For instance, any page referenced by a page with type L3 Page Table must itself have the type L2 Page Table.
This policy is applied recursively down to the L1 page table layer.
At L1 the invariant is that any data page mapped by a writable page table entry must have the writable page type.
By applying these policies, Xen ensures that all page-table pages as a whole are safe to be loaded into the cr3.
Similar requirements are also placed on other special page types, e.g., GTD/LDT pages.

The page type is allowed to be changed.
Xen enforces that the type of a page can only be changed when the type count is zero.
In addition, Xen also requires that every page type update only occur between writable and non-writable pages, as summarized in Figure~\ref{fig:page-type-updates}.

\begin{figure}[ht]
\centering
\includegraphics[width=0.45\textwidth]{image/background/page-type-updates.png} \\
\caption{Page Type Updates between writable pages and non-writable pages. (Writable pages are writable for both software and DMA
while non-writable pages are inaccessible to them.)}
\label{fig:page-type-updates}
\end{figure}


\subsection{DMA Validations}

\subsubsection{DMA Address Translation}
%citing: intel vt-d
The input/output memory management unit (IOMMU)~\cite{directio} is a memory management unit (MMU) that connects a DMA-capable I/O bus to the main memory.
Like a traditional MMU, the IOMMU maps device addresses (also called as I/O addresses) to physical addresses through a dedicated page table.
This technique is also known as DMA remapping.
The IOMMU page table that is created and maintained by the hypervisor in its own space, is able to restrict the access on a particular page by configuring the permission bits.
The hypervisor grants different access permissions for different page types, such as the writable pages are always allowed full access permissions, while the page-table pages are always inaccessible to any devices.
This is why when the page-type changes between writable page and page-table page, updating IOMMU page table is always necessary.

However, the DMA remapping always needs the page table walking, which is slow and inefficient.
To accelerate the translation speed, the I/O translation look-aside buffer (IOTLB) is introduced.
The IOTLB is used to cache frequently accessed page table entries.
By doing so, the IOTLB is very likely to be accessed, indicating that the physical address of a queried DMA address will be immediately fetched through the IOTLB path (Figure~\ref{fig:subfig:a}).
If unlikely the IOTLB miss occurs, the DMA remapping still can go the slow page-table path to get the physical address (Figure~\ref{fig:subfig:b}).
To achieve a better I/O performance, the DMA remapping should avoid taking the page-table path as far as possible.

\subsubsection{Negative Impacts}

\begin{figure}[!t]
    \begin{subfigure}{0.5\textwidth}
        \includegraphics[width=1\textwidth]{image/background/DMA-IOTLB-translation.png}
        \caption{\centering IOTLB Path.}
        \label{fig:subfig:a}
    \end{subfigure}%
    \vfill \vfill \vfill \vfill
    \begin{subfigure}{0.5\textwidth}
        \includegraphics[width=1\textwidth]{image/background/DMA-pt-translation.png}
        \caption{\centering I/O Page-Table Path.}
        \label{fig:subfig:b}
    \end{subfigure}
    \caption{IOTLB path is orders of magnitudes faster than I/O Page-Table Path.}
    \label{fig:dma-add-trans}
\end{figure}


\subsection{Long Execution Path}\label{sec:longpath}
The long execution path issue refers to the execution paths for installing/uninstalling a the guest page table page.
In the current design, installing/uninstalling a page table page has to invoke the complex memory allocators and performance both software (i.e., page table) and DMA validations.
%In addition, the DMA validations always lead to additional IOTLB flushes, which would reduce the DMA address translation speed and may consequently lower the I/O performances.
In addition, such installation and uninstallation events are frequently happened in the system, e.g., the process creation and exits will result in many page-table page installations/uninstallations.
Thus, it is necessary to shorten the execution paths to save CPU usage, benefiting 

%As the uninstallation process of the guest page table page is the reverse to the installation process. In this section, we only present the details of installation.
\subsection{Additional IOTLB Flush Issue}
The dependency between the guest page table and the IOTLB is very subtle.
To fully understand the dependency, we need to know the details about how the paravirtualized hypervisor protects itself through guest page table and IOMMU.
Specifically, we explicitly describe 1) paravirtualized MMU mode, and 2) IOMMU and DMA address translation.

Based on the observations and deep analysis, we know that there are seven page types, and the updates among them always trigger additional IOTLB flushes.
In particular, segment descriptor pages are rarely updated, as they are treated as almost const structures.
However, the page-table pages are frequently updated from/to writable pages.
These updates that are driven by the process creations and exits are frequently triggered in the whole life cycle of a running system.
Thus, they are becoming the main source for contributing the additional IOTLB flushing.

The additional IOTLB flushes are likely to let the DMA address
translations take the slow and inefficient page-table path,
instead of taking the fast and efficient IOTLB path (Figure~\ref{fig:dma-add-trans}), which inevitably lowers the
speed of the whole DMA transfering, especially for the high performance devices.


In brief, we summarize all these into three key points, which are listed as follows:
\begin{enumerate}
\item (O1) Each page-type change triggers the invalidation of at least one IOTLB entry.
\item (O2) The main source of causing IOTLB flush is the page-type changes between writable pages and page-table pages (see figure~\ref{fig:pro-ill}).
\item (O3) The additional IOTLB flushes inevitably have negative impacts on the I/O performance of the peripheral devices.
\end{enumerate}

\begin{figure}[ht]
\centering
\includegraphics[width=0.5\textwidth]{image/background/problem-illustration.png} \\
\caption{DMA access has to walk along the I/O Page-Table path, which is much slower than the IOTLB path due to the frequent page type updates between writable pages and page-table pages.}
\label{fig:pro-ill}
\end{figure}


%\section{Design Rationale}\label{sec:rationale}

This section analyzes several challenges of ROP attack detection, and introduces the main ideas and design decisions of ROPTerminator.

\mypara{When to perform ROP checking}
The timing of the ROP checking is critical, since it will directly affect the accuracy and performance.
As in G3, our ROPTerminator aims to protect any unmodified applications. Thus, we cannot leverage binary instrumentation to insert code in critical execution paths of the protected application, to trigger our ROP checking. It is obviously not efficient to monitor every operation of the application. while sampling in a constant or random frequency would improve the performance but is likely to introduce high false negative rate, since the ROP attacks may happen between two sampling points. %Thus, we need to choose an appropriate checking point to involve the ROP checking algorithm with low performance overhead. %can not give us the confidence that it can catch the ROP attack as long as it is really happened.

In this paper, we propose to monitor the application through a sliding window, in which the application code is set executable by ROPTerminator,
while the rest of code is non-executable. The execution flow, when jumping out of this window, will automatically trigger our ROP checking logic.
This monitor window design is reasonable because of the following two reasons. (1) The window size is small comparing with the large code base that is needed in launching a meaningful ROP attack. The attack cannot be finished within a window, so it will always trigger multiple instances of ROP checking. This is non-bypassable, because a user-space ROP attacker cannot disable the monitor window set by ROPTerminator from kernel-space.
(2) The performance overhead of the monitor window introduce is quite small, since there is no intervention for the operations running within the monitor window. In addition, the checking procedure is not triggered frequently due to the temporal and spatial locality feature of the execution~\cite{ulk}.


\mypara{How to defend against ROP attacks}
Based on different characteristics of ROP attacks, researchers have proposed several approaches to detect ROP attacks.

Some of the previous approaches observe that the traditional ROP attacks (ret-based ROP) violate the call-ret pairs. Thus, they proposed several approaches ~\cite{ropdefender,drop} to check if there is call-ret violation in runtime. Obviously, such approaches are able to detect the ret-based ROP attacks, but fail to handle the jmp/call-based ROP (conflicting our G1). Moreover, these methods incur unbearable performance overhead, because the checking procedure will be involved with every execution of the ret instruction.
Given that all ROP attacks violate the Control Flow Graph (CFG) and Call Graph (CG), some approaches~\cite{cfi,cflocking} are proposed to verify the control flow integrity in runtime. Note that such information is extremely difficult to extract \emph{completely} and \emph{accurately} from the (stripped) binary code. They have to rely on source code access and/or customized compiler tool-chain, thus violating G2.
Some other approaches~\cite{smashing,g-free} aim to remove gadgets from the binary code. However, such approaches usually can not give a clear answer on whether the remaining non-removed gadgets would be sufficient for constructing meaningful ROP attack. In addition, some of them may need source code and customized compiler tool-chain.
%Some approaches~\cite{ilr,binary-stirring} attempt to randomize the application in different level (e.g., in the instruction level) to increase the bar for the adversary launching ROP attacks. However some attacks sill can

In this paper, we propose a new ROP detection method, which relies on the payload detection and alignment filtering. The method is based on the observations:
\begin{enumerate}
\item \emph{A payload is necessary for a ROP attack to chain the selected gadgets}.
\item \emph{The unaligned gadgets are usually used as the critical elements in a gadget chain.}
\end{enumerate}

%\emph{\textbf{A payload is necessary for a ROP attack to chain the selected gadgets and the unaligned gadgets are usually used as the critical elements in a gadget chain}}.

The alignment filtering attempts to filter the ROP attacks using the unaligned gadgets. ROPTerminator checks each taken branch recorded in the LBR registers with the branch-alignment database generated in the pre-processing phase. Once an unaligned branch record is detected, ROPTerminator will kill the application with no doubt. \emph{The meaning of the alignment filtering is that it increases the bar for the adversary to launch a ROP attack because they have to give up using all unaligned gadgets.}

The payload detection aims to identify ROP payload from the application stack by constructing a gadget chain. If there is no detected gadget chain (e.g., the length of the constructed chain is under a threshold), ROPTerminator decides that no ROP payload is found at this round, and resumes the application execution. Otherwise, ROPTerminator puts the suspicious gadget chain to a global buffer and trigger the feedback confirmation mechanism to verify whether this chain will be executed in future. Once this gadget chain appears in the execution trace, ROPTerminator reports that a ROP attack is detected.

Note that the branch filtering step aims to ensure that there is no ROP attacks executing with unaligned gadgets in the past, and the payload detection attempts to identify all ROP attacks that may perform in the near further.

%In this paper, we propose a new payload checking method to detect ROP attacks. ROPTerminator relies on the information of branch instructions (e.g., branch type, alignment) to identify potential gadget pointers in application stack. We then attempt to construct a gadget chain by analyzing the stack-manipulation instructions (e.g., pop \%eax) in the identified gadgets. If the gadget chaining process terminates without success (e.g., the length of the constructed chain is under a threshold), ROPTerminator decides that no ROP payload is found at this round, and resumes the application execution. Otherwise, ROPTerminator puts the suspicious gadget chain to a buffer and trigger the feedback confirmation mechanism to verify whether this chain will be executed in future. Once this chain is recorded in the execution trace (Section~\ref{sec:lbr}), ROPTerminator reports that an ROP attack is detected.

%The branch and stack-manipulation instruction information could be reliably extracted from the application binary, by using the recursive descent disassembler (e.g., IDA Pro) or the linear scan disassembler (e.g., objdump). The disassembled results can be verified by the technique used in ~\cite{static-disassembly}.

%and the observation of the ROP gadgets (Section~\ref{sec:background}). together with the branch-alignment checking. The basic idea comes from the following observations:

%\begin{enumerate}
%\item The density of the gadget pointer is high in the ROP payload, while it would be quite low in the normal stack.
%    In a stack, the data that point to code space usually are return values and function pointers, but the pointed code sequences are not treated as gadget in most cases.
%    Note that we have proposed a new method to defend against the density dilution behaviour (in Section~\ref{sec:algo}).
%\item The taken branches are aligned in the normal execution, but some of them may be unaligned in the ROP attack.
%    Thus, once we observe that there is an unaligned taken branch, we can ensure that there is a ROP attack executing.
%\end{enumerate}

%This information is relatively easy to get the branch and alignment information comparing with getting CFG from the binary code.
%In our case, we only need to know if a branch instruction is aligned or unaligned, while the CFG requires more information, e.g., the predecessor and successor of a block.
%To extract the alignment information, we can use the recursive descent disassembler (e.g., IDA Pro) or the linear scan disassembler (e.g., objdump). In addition, the results can be verified by leveraging the technique used in ~\cite{static-disassembly}. %Note that such operations also can be done in the pre-processing phase. Thus, we have enough time to verify the correctness of the result. Getting the historical branch records from LBR registers, we can verify if a branch is unaligned in constant time.


%The payload checking needs to verify if the pointed address is a gadget, which implies that 1) the pointed address is in the code region; and 2) there is an indirect branch instruction in the following several instructions. Given that the gadget is shot, the verification is quite fast. In fact, the relative position of the gadget in a binary code/library is fixed. Thus, we can parse the gadget information in the pre-processing phase, and store it in a database. In the runtime, we can get the result in \emph{constant time} by looking up the database table.


%The second challenge is how to collect abundant semantic information for ROP checking. The information is mainly from the source code and the runtime information. The source code can produce lots of useful information, e.g., we can get the complete and accurate control flow graph and call graph for facilitating the ROP checking procedure. Unfortunately, we have no attempt to use the source code according to our goals. Thus, we have to carefully collect information in runtime. Note that the runtime information is accurate but may be incomplete, meaning that it may not induce false alarms.

%To collect runtime information, researchers usually choose to insert some code around the critical execution path by leveraging binary rewriting technique. However, we have to give up this method since we have no attempt to modify the binary code of the protected application.

%\mypara{How to minimize the false alarms}

%The ROP detection phase may introduce false positive and false negative. Specifically,
%The alignment filter mechanism will never introduce false positive, but it may miss catching the ROP attacks that only use aligned gadgets. Fortunately, the payload detection algorithm may be able to handle the missed ROP attacks.%, which will be handled by the payload detection algorithm.

%The payload detection algorithm may introduce false positive, since the adversary can manipulate the input to the target application to intentionally fill a stack buffer with data that contain a fake gadget chain. It may lead to Denial of Service (DoS) attack, if ROPTerminator attempts to stop the application under this situation. The feedback mechanism can eliminate such cases .

 %the ROP checking algorithm. Essentially, the feedback mechanism is to double check whether the gadgets pointed by the \emph{suspicious gadget pointers} are executed. If they are really executed, we then can make sure that it is a real ROP attack.

%It is extremely hard for the adversary to bypass our verification mechanism.
%The adversary may attempt to bypass our verification mechanism by challenging certain observations:

%The adversary may attempt to launch \emph{payload dilution} attack to bypass the payload checking. Specifically, the payload dilution attack is to dilute the density of the gadget pointer by adjusting the $ESP$ position, e.g., the gadget \textit{mov $0x2c$,\%esp; pop \%ebp; ret} attempts to fetch the next gadget pointer by moving $ESP$ up $48$ bytes. We have proposed a novel algorithm to get the real density by smartly calculating the position of $ESP$ (see details in Section~\ref{sec:algo}).

%The adversary may launch the \emph{stack pivoting} attack~\cite{jit-code-reuse} to confuse the ROPTerminator gadget linking process. Essentially, the stack pivoting attack is putting the payload into another memory region (e.g., a heap buffer or the global data region), and manipulating the stack pointer to point to that region. If ROPTerminator still checks the original stack region, the pivoted payload remains unfound. In fact, ROPTerminator can defend against such attacks by checking whether the $ESP$ is still in the stack region. Note that ROPTerminator in the kernel space is able to acquire the stack region information.

%The adversary may leverage the \emph{gadget gluing} attack to change the format of the gadget and bypass ROPTerminator checking. Normally, the direct branch should not appear in a gadget. However, in certain special cases, the code sequence that a direct branch points to may contain a gadget. The adversary may attempt to glue the direct branch to the gadget to form a new gadget. To defend against this attack, ROPTerminator can choose to follow one hop to verify whether there is a gadget at the destination of the branch.


% density checking as the second defense line to
% alignment checking as the last


%Note that the adversary can not conceal the alignment information of the taken-branches since they can not modify the LBR registers.

%There is almost no false negative, since we are able to use the lower bound of the density of gadget pointer as the threshold. Note that even the lower bound of the density is also extremely high according to our density calculation algorithm (i.e., the smarter counting algorithm in Section~\ref{sec:algo}). Thus, it would not produce many false positive. In addition, the feedback mechanism is able to filter the positive cases.


%The first challenge is due to the absence of the side information. The side information may provide a lot of useful information, such as the complete and accurate control flow graph and call graph, which can be used to accurately verify a ROP attack. Missing such information may lead to the inaccuracy (i.e., false positive and false negative) of the ROP verification. Consequently, it may result in the high performance overhead since we need to collect more runtime information to increase the accuracy.


%The second challenge is to achieve low performance overhead without modifying the binary. Without the inserted code raising a signal to trigger the ROP checking, it usually implies to monitor every step of the execution of the protected application. Obviously, this solution will suffer high performance overhead.

\section{System Overview} \label{sec:overview}
\begin{figure}
 \centering
\includegraphics[width=\columnwidth]{"image/arch2"}
\caption{\textbf{The architecture of \name.} The shaded square represents a protected application.}\label{fig:arch}
\end{figure}
\subsection{System Architecture}\label{sec:phases}

Figure~\ref{fig:arch} shows the high-level architecture of the \name process.
The \name has an offline analysis phase to extract instructions from the binary and generate Branch \& Gadget (BG) database that records the branch and gadget information.% (shown in Figure~\ref{fig:value-table}).
The runtime system consists of the operating system kernel, the \name itself, one protected application (i.e., the shaded one), and other unprotected applications.
The \name module is a dedicated loadable kernel module that monitors the execution of the protected application with the attempt of defending against ROP attacks.
\name can be automatically loaded during system boot-up, or dynamically activated by the end users.
Once activated, \name only collects information (e.g., stack, execution trace) and the BG database about the protected application.
\name is able to protect any unmodified application binary, and can simultaneously monitor multiple applications.

\mypara{General Workflow}
The general workflow of \name is divided into two phases,
an offline pre-processing phase, and a run-time detection phase.
In the {\em offline pre-processing phase}, the \name pre-processor extracts all
branches information (position and type) and instruction alignment information
of the selected applications and the shared libraries they depend on.
The \name pre-processor also extracts the potential gadgets from the binary,
and analyzes the stack manipulation behavior (e.g., pop instruction moves up the top of the stack 4bytes) of those gadgets.
The pre-processor then leverages the branch, alignment and gadget information
to construct a database that will be loaded in run-time phase for the ROP checking.
Note that the databases of the shared libraries (e.g., libc) are only
generated once, and can be re-used for protecting different applications.

During the {\em run-time detection phase}, \name loads the databases of
the monitored applications and their libraries to its memory space,
and sets a monitor window (see Section~\ref{sec:monwin}) to monitor the each protected application.
If the application executes in the monitor window, no ROP checking is invoked,
and the execution is in the native speed. Once the execution flow
jumps out of the monitor window, \name suspends the application,
attempts to identify gadget pointers from the current application's stack.
If these gadget pointers can be linked to form a suspicious ROP payload,
\name will record the suspicious payload and activate the feedback mechanism to double check it later.
Once it is confirmed, \name kills the victim application and reports the detected ROP attack.
Due to the temporal and spatial locality of the application,
such checking procedure is triggered infrequently.


\subsection{ROP Detection}\label{sec:ropdetect}
\subsubsection{Monitor Window}\label{sec:monwin}

\begin{figure}
 \centering
\includegraphics[width=\columnwidth]{"image/monwin"}
\caption{\textbf{The monitor window and its update.} In this example, the size of the monitor window is 2 pages. When the execution flow reaches to line 2, the monitor is \emph{(page\_b, page\_c)}. Later it is updated to \emph{(page\_a, page\_b)} when the execution flow reaches to line 5. Note that the pages in the monitor window may not be continuous.}\label{fig:monwin}
\end{figure}

A monitor window is an abstract concept, which refers to several executable code regions in the target applications. Note that the code regions in the monitor window may not be continuous. The purpose of the monitor window is to trigger
the ROP checking procedure with low performance overhead
and high accuracy.
During the execution of the target application, the code regions in the monitor
window is set executable by \name using page table permissions,
while the rest of the code area remains non-executable (Fig.~\ref{fig:monwin}).
Thus, the \name checking is only invoked when the execution of the application
jumps out of the monitor window. If the checking is passed and no ROP is found,
\name updates the monitor window by including the code regions newly accessed, and kicks certain previous
regions out, if needed, according to configurable monitor window size and
update policies. Note that the window size must be smaller than the minimum size of the
code-base that could provide gadgets for launching a meaningful ROP attack
(e.g., $20KB$, as is shown in~\cite{q}). Reducing the window size will increase the
frequency of the invocation of \name, and thus degrade the system performance.


\subsubsection{Branch Alignment Filtering}\label{sec:alignment}
The branch alignment filtering attempts to expose the ROP attacks by checking the unaligned gadgets.
As introduced earlier, the recent taken branches are recorded in the LBR registers,
and the alignment information is stored in the BG database.
Given the two inputs, \name verifies whether there is any unaligned branch in the historical results.
Even if only one unaligned branch is caught, \name will ensure the existence of a ROP attack, and terminate the victim application.

The branch alignment filtering introduce \emph{no} false positive
since normal execution does not use the unaligned instructions.
Thus, the adversary has to leverage aligned gadgets to form the ROP gadget chain,
and it may lead to the failures of ROP attacks due to the lack of critical gadgets.


\subsubsection{Payload Detection}\label{sec:payload}
The payload detection is to examine the application stack and attempts to
identify a ROP payload. \name treats each stack entry (e.g., 4bytes on x86 platform) as a potential
gadget pointer. If the region that it points to (1) is a code region of
the application itself or its shared libraries, and (2) can be disassembled
to a sequence of instructions ended by an indirect branch instruction, and
is shorter than a {\em gadget length threshold}, \name treats it to be a gadget. Setting this threshold too low,
it would cause \name to neglect long gadgets, and increase our false negative
rate. If it is too high, the \name might mis-identify valid code sequences as
gadgets. Recall that a gadget roughly has 2-5
instructions~\cite{tc-ret2lic}. We set the threshold as 6 in this paper.

\name then attempts to link the detected gadget to form a potential
ROP payload. Note that gadgets may contain instructions to adjust
the position of the stack pointer, e.g., the
gadget \emph{pop \%eax, pop \%ebp; ret} will increase the stack
pointer $ESP$ by 8 bytes, and return to the next pointed address.
Such \emph{stack-manipulation} technique is used by the adversary to link the gadgets
in the payload. To construct this gadget chain, \name analyzes all
instructions in the identified gadget to figure out the stack adjustment
and locates the next gadget pointer in the stack. If the number
of the linked gadgets reaches a {\em payload length threshold}, \name
successfully detect a suspicious ROP payload. If the gadget chain breaks
at certain stack entry (e.g., pointed to data region, or code regions
that have no gadget), the \name will proceed to the next stack entry, i.e.,
the stack pointer increased by four bytes, and retry the gadget chaining
process until we reach the {\em retry number threshold}.
If number of the detected gadgets (linked plus unlinked) is still below
the {\em payload length threshold}, \name aborts this round of check,
otherwise, \name stores all detected gadget pointers into a buffer
and enables our feedback confirmation mechanism.

The {\em payload length threshold} represents a trade-off between false
positive rate and ROP checking overhead. For example, the higher
this threshold is, the less possibility that \name mis-identifies
a valid stack frame trace as a ROP payload, but the more number
of stack entries \name need to process. Fortunately, the gadget chain identified
in the normal application stack is quite short or none.
\name introduces retry mechanism to reduce the false
negative rate of gadget linking, since the gadget links might be
broken at the pointers that lead to a gadget longer than our {\em gadget
length threshold}.


\subsubsection{Feedback Confirmation}\label{sec:feedback}
This step aims to further reduce the false positive produced in the payload detection step.
For example, a large data buffer of the application may store multiple values
that accidentally point to gadget-like code area.
An adversary may take advantage of this and launch a Denial of Service (DoS) attack,
by manipulating the input to the target application and filling many gadget pointers
to the stack. Our feedback mechanism is based on the key observation:
the pointed gadgets should be executed in the near further if it is really a ROP payload.
Specifically, after payload detection process, \name saves the gadget pointers and
allows the application to continue running. When \name is invoked next time, it
reads the historical branch results in the LBR registers,
compares them with saved gadgets pointers. If the saved gadgets pointers appear
in the LBR, \name can ensure that such gadgets are really executed.
This feedback buffer might be used at multiple invocations of \name, if
it is continuously appended with newly found gadget pointers.
However, the feedback buffer
might be refreshed totally, if the next payload detection process detects a different
gadget chain. The feedback mechanism will be turned off, if there is no gadget
chain found in the following payload checking invocation.
Note that the feedback mechanism, while reduces false positive, postpones
the detection of the ROP attacks, until it starts executing.
%Section~\ref{sec:imple} provide further details of our feedback confirmation mechanism.
%This indiates that either the branches in the original ``gadget chain''
%is not executed, or they have finished executing for a long time so the records
%in the LBR is cyclically rewritten by latest happend branch. they do not appear in the LBR branch records until next ROP alarm, we simply update the gadget pointer table and continue the next phase of checking.


\subsection{ROP Checking Algorithms} \label{sec:algo}

\begin{figure}[!ht]
 \centering
\includegraphics[width=0.99\columnwidth]{"image/algo"}
\caption{\textbf{The algorithm for ROP checking.} The algorithm is invoked when
the application execution flow jumps out of the monitor window (i.e., $IP_d$ is outside).}
\label{fig:algo}
\end{figure}

This section introduces how we apply the techniques in Section~\ref{sec:ropdetect}
in each run-time \name invocation to detect ROP attacks.
Once triggered, ROPTerminator can reliably acquire the stack information (e.g.,
stack range, stack pointer) of the target application. \name knows
the virtual memory mapping of application and all loaded shared libraries, and loads all relevant BG databases. We then introduce
the major steps in our run-time detection mechanisms, and defer
the implementation details to Section~\ref{sec:implement}.

As shown in Figure~\ref{fig:algo}, the first step of the run-time detection algorithm
is {\em branch alignment filtering}, which aims to check if there are
unaligned gadgets in the historical branch records. For the normal executions,
the branch records in the LBR should be aligned, both the source and the destination
address. But in a ROP attacks, the adversary may use some unaligned gadgets to
operate certain behaviors.
This checking is able to filter out all ROP attacks that use the unaligned indirect branch instructions.
We analyze all ($568$) applications under directory $/bin$ and $/usr/bin$, and the
results of our statistical analysis (see Table~\ref{tab:unaligned}) show that generally 62\% indirect branches are unaligned. Furthermore, in certain applications, the rate number can even reach to 90\%.

\begin{table}
  \centering
  \begin{tabular}{|c|c|c|c|}
     \hline
     % after \\: \hline or \cline{col1-col2} \cline{col3-col4} ...
      & Min & Avg & Max \\ \hline
      Unaligned / Total & 23.2\% & 62.0\% & 90.0\% \\
      %Unaligned / Total & 23.2\% & 57.2\% & 90.0\% \\
     \hline
   \end{tabular}
  \caption{The statistical result of the unaligned/total indirect branches. These numbers are got through analyzing $568$ applications under directory $/bin$ and $/usr/bin$.}\label{tab:unaligned}
\end{table}

The step 2 is {\em feedback Confirmation}, which is disabled by default, unless it is explicitly enabled by the payload feedback mechanism triggered by step 3. Specifically, if it is disabled, \name does nothing and directly jumps to step 3.
Once it is enabled, \name checks if the suspicious payload saved before is taken. 
Specifically, for each gadget pointer in the suspicious payload, \name verifies if it is in the LBR registers.
If the number of the matched gadget pointers reaches to the payload length threshold,
\name confirms that a ROP attack is in action, and acts accordingly (e.g.,
stop the victim application).

In Step 3, we perform {\em payload detection} in the application stack,
starting from the branch instruction that jumps
out of the monitor window. Recall that in the offline pre-processing phase, we pre-process all gadget information,
and stores it with branch information in databases. Given the destination address
of the branch, we can quickly find out the corresponding database entries. The entry
basically tells \name about whether there is a gadget at this address, and what stack-manipulation
action will that gadget perform. This significantly improves ROPTerminator's run-time performance,
by offloading all gadget identification operations to the pre-processing phase.
With the BG database information, \name executes the payload checking algorithms
presented in Section~\ref{sec:payload} to find out a suspicious ROP payload (either fully linked,
or broken at certain stack position). If there is no suspicious payload found, the \name
claims no ROP detected at this round and perform {\em Monitor Window Update} before resuming
the protected application.
If there is a suspicious payload, \name generate a payload feedback by saving all detected gadget pointers in a preserved buffer, and enables the feedback confirmation checking (i.e., the step 2).



\mypara{Optimization}
To accelerate the run-time process, we propose an optimization for this detection
algorithm, {\em Branch type Filtering}. This filtering mechanism is added into our
algorithm, combined with {\em branch alignment Filtering} step.
With this mechanism, \name checks a few latest recorded LBR branches
to evaluate the existence of ongoing ROP attack. If it indicates that there is no ROP attack, \name skips this round of {\em
ROP Checking} and resumes the application execution.
The filtering is based on the observation that:
\emph{The gadgets are linked through the indirect branch instructions, rather than the direct branch instructions.}
%We verify whether the ratio of the direct branches in LBR recorded branches are very high, and the rest of the indirect branches are either not gadgets or not linked together. \name will perform payload checking only when the verification fails.
Thus, if we find a direct branch in the last $k$ (e.g., 3) records, we can skip this round of checking.
This optimization will greatly improve our performance,
but with the risk of skipping $0$ - $(k - 1)$ rounds of payload checking
when a ROP is really in action, e.g., the round right before the
ROP attack starts. Given that a meaningful ROP attack would need
many gadgets and always trigger multiple rounds
of checking, the risk is quite a little low. In addition, the end users can adjust the number $k$ to manage the possibility of the risk according to the real demand.

\section{Implementation} \label{sec:implementation}
\yueqiang {
implementations:
1) hooks for on-demand enable
2) locks for mechanism
3) data structures in mechanism
4) threshold selection for cache free? or this discussed in design
5) }

Based on Xen version 4.2.1 and guest kernel version 3.2.0-rc1, \name implements the access control related to cache flag as well as management of page-table cache.

\subsection{Cache Flag}

From a 32-bit field related to page type information, \name was planning to use a redundant bit to represent the cache flag in order to be compatible with its original implementation. However, each bit in the upper 9-bit of the field has its specific use and the lower 23-bit serves as the reference count of current page type, representing at most ($2^{23}-1$) reference counts of the page type (see figure ~\ref{fig:subfig:a}). \name enforces a stricter rule to prevent count overflow, as cache flag occupies the top bit of the 23-bit and the maximal reference count is limited to ($2^{22}-1$), which affects Xen little (see figure ~\ref{fig:subfig:b}). Actually, Xen hypervisor is functioning well with the flag bit since the reference count of a page type during runtime cannot reach at ($2^{22}-1$). On top of that, \_\_get\_page\_type() is the critical function to be customized in order to check if a machine page has the flag bit.

\begin{figure}
\centering
\subfigure[Existing Representation]{
\label{fig:subfig:a}
\includegraphics[width=0.5\textwidth]{image/implementation/field-of-page-type-info.png}}
\hspace{1in}
\subfigure[Cache Flag Representation]{
\label{fig:subfig:b}
\includegraphics[width=0.5\textwidth]{image/implementation/field-of-page-type-info-with-cache-flag.png}}
\caption{Field of Page Type Info}
\label{fig:PGtime} %% label for entire figure
\end{figure}

\subsection{Management of Page Table Cache}

In the PV setting, guest OS is required to work in PAE (i.e., Physical Address Extension) mode. As a result, \name maintains three levels of cache pools, each of which is essentially a structured single-linked list caching the guest physical addresses of pages. For the top two levels used for caching PGD (i.e., Page Global Directory) and PMD (i.e., Page Middle Directory), \name obtains the physical addresses directly from their linear addresses. While the bottom PT (i.e., Page Table) is located in the High Memory, the mapping between its linear and physical addresses is not stable. \name acquires the physical addresses of cached PT pages from the corresponding page-info structures.

Every list supports operations of both removal and insertion. Removal exists within the cache allocating function, which fetches a cached page also from the top. Insertion exists within the cache freeing function, which inserts a cached page onto the top of the list. Also, \name maintains two global operation counters, namely num\_in\_use and num\_in\_pool, each logging the invocation times of removal and insertion, respectively. They are required when cache freeing function needs to free cached pages into the buddy system. Every pair of cache allocating and freeing functions is used to hook existing page-table related functions. By this way, whenever OS creates or destroys any process, it always interacts with the caches. Also note that every cache pool and operation counter are shared data resources and their related operation code are critical sections. \name makes use of locks to ensure exclusive resource-access in the multi-processor setting. More specifically, Every level of page-table cache and all its operation counters share a spin lock and irrelevant operations should be removed out of the critical section, especially those that are time-consuming.

Cache mechanism begins its lifecycle either automatically with system booting up or dynamically during system runtime. \name implements a new system call to provide an interface for users to activate the cache in an on-demand way. In the system call, a global boolean variant is defined, which is initialized as false, indicating that cache is not enabled. And users can assign the variant as true to enable the cache. If the cache mechanism is running while thresholds of the two factors are met, cache freeing function is responsible for freeing pages into the buddy system , thus a corresponding hypercall should be invoked with a single-linked list of cached addresses of pages as parameters. The hypercall is a uniform interface for every level of cache freeing function, simplifying the modifications both to guest OS and Xen. As a reply to the hypercall, Xen mainly clears the flag bit of specified pages, creates entries in the I/O page tables and flushes entries of IOTLB. It can be concluded that inappropriate thresholds will lead to the flushes of IOTLB. Because of that, \name develops an interface for users to modify the default values of both thresholds in order to reduce IOTLB-flush. If available memory in the system is too small in extreme cases, users are provided with another interface to manually free all pages in cache into the buddy system, thus relieving system's pressure while badly affecting IOTLB.
 



\section{Evaluation} \label{sec:eva}
We have implemented the prototype of \name on our experiment platform.
Xen version 4.2.1 is the hypervisor while the guest VM (i.e., Dom0) is Ubuntu version 12.04 with Linux kernel version 3.2.0.
%The \cache introduces $350$ SLoC in the Linux kernel and the \module adds $166$ SLoC in Xen.
\name added or changed $350$ SLoC in the Linux kernel and $166$ SLoC in Xen, respectively.
To fully evaluate the performance and its effects on the whole system, we measured the \name in both micro-benchamrks (e.g., the frequency of IOTLB flushes, the execution time of page table allocation) and macro-benchmarks (e.g., \emph{SPECINT}, \emph{netperf} and \emph{lmbench}).

\subsection{Experiment Setting}
The experiment platform is a LENOVO QiTian M4390 PC with four CPU cores (i.e., Intel Core i5-3470) running at 3.20 GHz.
We enable the Intel VT-d feature through BIOS and grub configuration file. The IOMMU supports queue-based invalidation interface in the granularity of page invalidation.

%In the original design of Xen, page table (de)allocations will give rise to page type updates, upon which the function iotlb\_flush\_qi will be invoked to flush corresponding IOTLB entries. Thus, a global counter is placed into the function body to record invocation times of the function and then an average counter per minute is calculated as a frequency of IOTLB-flush. When the IOTLB-flush stays at zero level, it means that no page table is (de)allocated, indicating that no process creation/exit occurs then. In a nutshell, there exists a mutually positive effect between a process creation/exit and IOTLB-flush.

\mypara{Workload Emulation}
In order to allow us to repeatedly measure the effects of the \name on 1) page table allocations and deallocations, and 2) the IOTLB flushes, we use a stress tool to explicitly emulate a heavy workload with many short-time concurrently-running processes.
Specifically, the tool periodically launches a browser (i.e., Mozilla Firefox 31.0 in the experiment), continuously opens new tabs one by one, and terminates the browser gracefully.
The purpose of these operations is to frequently create and terminate a large number of processes, leading to many page table allocations and deallocations.
The frequency can be configured. In our experiment setting, there are $542$ processes created and exited per minute.
In order to avoid the browser occupying too much memory, we terminate it in every 5 minutes.
At this moment, the memory usage of the browser reaches to $284.1$ MB on average.

%Both micro- and macro-benchmark measurements are performed under this environment. in which micro tests are utilized to evaluate the frequency of IOTLB-flush, CPU usage and memory size while macro-benchmarks give an assessment on overall system performance.

\subsection{Micro-Benchmarks}
The micro-benchmark measurements are to evaluate the frequency of the IOTLB flushes, as well as the CPU usage and the memory usage of the \name.
For each measurement, there are two control groups and one baseline/normal group.
In the baseline group, we run the workload emulation in the guest VM with default settings, without enabling the \name mechanism.
On the contrary, the two control groups are: 1) the \prename group, where the \name is enabled before the workload emulation starts and 2) the \dynname group, where the \name is dynamically enabled (e.g., five minutes after the workload emulation launches). The \dynname group is to evaluate if 1) the \name is able to enter a stable state, and 2) how fast the \name is able to enter the stable state.

\begin{figure}[ht]
\centering
%\includegraphics[scale=0.55]{image/iotlbflush.png} \\
\includegraphics[width=0.5\textwidth]{image/micro/iotlbflush.jpg} \\
\caption{The frequency of IOTLB flushes.  In the \prename group the frequency is reduced from a very low level to zero within 1 minutes. In the \dynname group, the frequency drops sharply within two minutes from the high level to zero. Both control groups indicate that the \name could always enter the stable state (i.e., zero frequency).}
\label{fig:iotlbflush}
\end{figure}

\subsubsection{Frequency of IOTLB Flushes}
In this test, we aim to evaluate the effectiveness of the fine-grained validation on the addition IOTLB flushes.
We sample the frequency of the IOTLB flushes in 30 minutes.
The measurement results are illustrated in Figure~\ref{fig:iotlbflush}.
In the baseline group, the frequency of the IOTLB flushes is increasing in the first five minutes, and then keeps at a high flush rate until the test finishes.
In the \prename group, the flush frequency quickly decreases to zero level in about one minute and keeps at the level.
In the \dynname group, the flush frequency sharply decreases to zero when the \name is enabled. The \name roughly spends two minutes entering the stable state.
Thus, we can conclude that the fine-grained validation scheme is able to efficiently and effectively eliminate the IOTLB flushes introduced by the DMA validations.

%Y-axis represents the frequency of IOTLB-flush, corresponding to the time period (i.e., one minute) of x-axis for the first thirty minutes that the \emph{busy state} lasts. From this figure, frequency in the baseline group increases rapidly and remains stable five minutes later. By contrast, frequency in the pre-\name group drops to zero level in a very short time. It can be safely concluded that the fine-grained validation module proposed by \name does eliminate the IOTLB flushes caused by concurrent processes creations/exits. The dyn-\name group shows that the frequency also can be dropped to zero level very quickly even if the system is already in a \emph{busy state}.

\begin{figure*}[t!]
    \centering
    \begin{subfigure}[t]{0.5\textwidth}
        \centering
        \includegraphics[height=2.0in]{image/micro/PGDalloc.png}
        \caption{Execution time of L1 allocation is improved by 34\%.}
        \label{fig:l1a}
    \end{subfigure}%
    ~
    \begin{subfigure}[t]{0.5\textwidth}
        \centering
        \includegraphics[height=2.0in]{image/micro/PGDfree.png}
        \caption{Execution time of L1 deallocation is improved by 47\%.}
        \label{fig:l1b}
    \end{subfigure}
    \caption{The execution time of L1 allocation and deallocation. The \name in the \dynname group can quickly enter the stable state in 2 minutes.}
    \label{fig:PGDtime}
\end{figure*}

\begin{figure*}[t!]
    \centering
    \begin{subfigure}[t]{0.5\textwidth}
        \centering
        \includegraphics[height=2.0in]{image/micro/PMDalloc.png}
        \caption{Execution time of L2 allocation is improved by 38\%.}
        \label{fig:l2a}
    \end{subfigure}%
    ~
    \begin{subfigure}[t]{0.5\textwidth}
        \centering
        \includegraphics[height=2.0in]{image/micro/PMDfree.png}
        \caption{Execution time of L2 deallocation is improved by 22\%.}
        \label{fig:l2b}
    \end{subfigure}
    \caption{The execution time of L2 allocation and deallocation. The \name in the \dynname group can quickly enter the stable state in 2 minutes.}
    \label{fig:PMDtime}
\end{figure*}

\begin{figure*}[t!]
    \centering
    \begin{subfigure}[t]{0.5\textwidth}
        \centering
        \includegraphics[height=2.0in]{image/micro/PTEalloc.png}
        \caption{Execution time of L3 allocation is improved by 65\%.}
        \label{fig:l3a}
    \end{subfigure}%
    ~
    \begin{subfigure}[t]{0.5\textwidth}
        \centering
        \includegraphics[height=2.0in]{image/micro/PTEfree.png}
        \caption{Execution time of L3 deallocation is improved by 65\%.}
        \label{fig:l3b}
    \end{subfigure}
    \caption{The execution time of L3 allocation and deallocation. The \name in the \dynname group can quickly enter the stable state in 2 minutes.}
    \label{fig:PTEtime}
\end{figure*}

\subsubsection{CPU Usage Measurement}
��As there are three levels of the guest page table, and each level has its (de)allocation functions, e.g., pgd\_alloc and pgd\_free for the L1 level.
In order to clearly observe the changes of the CPU usage, we measure them separately.
We continuously measure them in 30 minutes, and calculate the average execution time of each function in each one minute.
Note that in the \dynname group, we enable the \name 5 minutes after the workload starts to run.


As shown in Figure~\ref{fig:PGDtime}, \ref{fig:PMDtime}, and \ref{fig:PTEtime}, the execution time in the \prename group, for a pair of allocation and deallocation of the three-level page table, from top to bottom are (622, 196), (424, 242) (260, 217) in nanoseconds, while in the baseline group, the corresponding execution time are (944, 366), (682, 313), (755, 627) in nanoseconds. Correspondingly, the improvements are (34\%, 47\%), (38\%, 22\%) and (65\%, 65\%), from top to bottom.
%putting together
Putting them together, the page table allocations and deallocation can be improved by 45\% and 55\% on average respectively.
Note that the \name in the \dynname group can achieve the same performance improvement as the one in the \prename group.
The transitional stage to the stable state only needs less than 2 minutes, as illustrated in Figure~\ref{fig:PGDtime}, \ref{fig:PMDtime}, and \ref{fig:PTEtime}.

\mypara{The Worst Case}
With \name enabled, guest kernel firstly goes through the \cache for page table allocations. If the cached semi-writable pages cannot satisfy the requirements (a.k.a, cache miss), the \cache would have to allocate writable pages using the existing memory allocators following the traditional paths,.
In this case, the execution time is the traditional execution path plus the path in the \cache.
As a result, the execution time of the page table allocation is even longer than that of the baseline (Figure~\ref{fig:overview} (a)).
However, the overhead introduced by the \name path is negligible, as the control flow will immediately return when the list is empty.
More specifically, the overhead consists of the function innovation provided by the \emph{pop} interface, related stack adjustment and the checking of the cache list and the page type.
Putting them together, the overhead are less than 20 instructions.
Fortunately, the worst case does not often occur. According to our observations in the \dynname group, the number of the worst cases is $348$, out of $198990$ allocation requests in 30 minutes.
Note that the page-table deallocation is always completed successfully in a constant time.

%\zhi{In the pre-\name group, the probabilities of cache miss are 9/550 (1\% PGD), 38/2200(1\% PMD), 322/6633(4\% PTE),respectively. In the dyn-\name group, the probabilities of cache miss are 7/540 (1\% PGD), 24/2160(1\% PMD), 320/6333(5\% PTE),respectively.}

%When cache is pre-enabled, the ratio between the total pages in cache and the total pages as page tables is approximately equal to 1:1, which indicates that the cache mechanism takes up a small memory percentage of the stress tool.

\begin{table}[!ht]
\footnotesize
\begin{center}
\begin{tabular}{|l|l|l|}
\hline
{\textbf{Levels of Page Table}} & {\textbf{\prename (\#)}} & {\textbf{\dynname (\#)}} \\ \hline
L1 & $5$  & $4$ \\ \hline
L2 & $26$ & $20$ \\ \hline
L3 & $145$ & $136$ \\ \hline
Total & $176$ & $160$ \\ \hline
\end{tabular}
\end{center}
\caption{Cache usages in both groups of \prename and \dynname are small, occupying 176 pages and 160 pages, respectively.}
%And the memory size of cached pages only takes up to 0.2\% of the tool's memory.
\label{tab:PGpool}
\end{table}

\subsubsection{Memory Usage Measurement}
Besides CPU usage, we also measured the memory consumption of the \cache.
To clearly see the page usage in each level, we measured the cached page numbers in three levels and results are listed in Table \ref{tab:PGpool}.
In the \prename group the \cache has $5$, $26$ and $145$ semi-writable pages ready for allocations, totally occupying $704KB$, which is quite similar to the \dynname group, where the cache has $160$ in total, occupying memory $640KB$. As a result, the \cache in both groups consume an insignificant memory usage, always less $1MB$.
%Also note that the conditions of freeing pages in cache in the dyn-\name group heavily rely on a specific scenario, which may not work for other situations. We will further discuss the conditions in future work.

%And the reason why the dyn-\name group has fewer cached pages than that of the pre-\name group is that a certain number of page table pages has been freed to the memory allocator before the \name is enabled.



\subsection{Macro-Benchmarks}
The purpose of the macro-benchmarks is to evaluate the effects of \name on the overall system.
All measurements are divided into two groups: 1) the baseline group with the default settings, and 2) the \name group with the \name enabled.

\begin{figure}[htp]
\centering
%\includegraphics[scale=0.55]{image/iotlbflush.png} \\
\includegraphics[width=0.4\textwidth]{image/macro/lmbench.png} \\
\caption{The execution time on average in the \name group is reduced by 11\%, 17\% and 21\%, from left to right.}
\label{fig:lmbench}
\end{figure}

\subsubsection{Latency Improvements in Process Creations and Exits}
Lmbench is a macro-benchmark tool for measuring the latency of the process creations and exits (i.e., fork+exit, fork+execve, fork+/bin/sh -c), shown in Figure~\ref{fig:lmbench}.
The Lmbench are configured using the default parameters, except for the parameters of processor MHz and memory range.
In our experiment platform, the CPU frequency is $3.2$ GHz, memory range is set as $1024$ MB to save measurement time.
As illustrated in Figure~\ref{fig:lmbench}, the processes of \emph{fork+exit}, \emph{fork+execve} and \emph{fork+/bin/sh -c} in the \name group costs $344$, $365$, and $1567$ in microseconds, $11$\%, $17$\%, $21$\% faster than the ones in the baseline group.
We believe that the improvement will significantly benefit the workloads that rely on many short-time temporary processes.

\begin{figure}[htp]
\centering
%\includegraphics[scale=0.55]{image/iotlbflush.png} \\
\includegraphics[width=0.5\textwidth]{image/macro/spec.png} \\
\caption{The differences among all the benchmarks are within 0.42\%, indicating that the \name has no negative effect on system performance.}
\label{fig:spec}
\end{figure}

\subsubsection{SPECINT}
SPEC CINT2006~\cite{specint} is an industry standard benchmark intended for measuring the performance of the CPU and memory.
In our experiment, the tool version is SPECint 2006 v1.2, which has $12$ benchmarks in total and they are all invoked with the configuration file \emph{linux64-ia32-gcc43+.cfg}.
All measurement results are listed in Figure~\ref{fig:spec}.
Among the benchmarks, the time differences between the baseline group and the \name group are very small, e.g., the maximum difference is within 0.42\%, which is produced by the \emph{483.xalancbmk}.
All of them produce the same or a little better performance results, indicating the \name has no negative effect on system performance.

\subsubsection{I/O Performance Measurements}
As we know, IOTLB is used to accelerate the DMA address translation to achieve better performance for I/O devices.
Therefore, if there are many IOTLB misses caused by frequent IOTLB flushes will introduce negative effects on the  I/O performance.
However,  rIOMMU~\cite{malka2015riommu} claims that the overhead caused by walking the IOMMU page tables due to IOTLB misses is so negligible that cannot be measured in the \emph{netperf}, because the main latency induced by I/O interrupt processing and the TCP/IP stack is several orders of magnitude larger than that of walking the page tables.
In addition, Nadav Amit et al.~\cite{amit2012iommu} also has similar statements.
% that the IOTLB misses cannot be observed under regular circumstances, since the I/O memory (un)mapping operations consume much more time than that of the corresponding DMA transaction.

In this paper, we test the network I/O and disk I/O performances under regular circumstances.
%, and use these experiments as the revisiting of the problem between the IOTLB misses and I/O performance.
We measured the network I/O using \emph{netperf} tool. Specifically, we have two machines that are directly connected through the Ethernet cable.
The client on one machine set a bulk of TCP packets to the server on another machine. Note that emulated workload is also enabled on the client machine to trigger IOTLB flushes.
The sending buffer is $16KB$ and the test lasts 60 seconds.
The measurement results are listed in Table~\ref{tab:netperf}.
By comparing both values of $\mu$ and $\sigma$, we find that there is no detectable negative effect on the network I/O.
Similarly, we also test the disk I/O using lmbench. The results indicate that the disk I/O speed remains the same.
The above two experiments also as new evidences to support the observations in ~\cite{amit2012iommu, malka2015riommu}.

\begin{table}[!ht]
\footnotesize
\begin{center}
\begin{tabular}{|l|l|l|}
\hline
{\textbf{Throughput (\emph{Mbps})}} & {\textbf{\name}} & {\textbf{Baseline}}    \\ \hline
Range & $87.880-88.010$ & $87.880-87.950$ \\ \hline
Arithmetic Mean ($\mu$)  &  $87.926$ & $87.913$ \\ \hline
Standard Deviation ($\sigma$) &  $0.028$ & $0.021$ \\ \hline
\end{tabular}
\end{center}
\caption{The netperf results of network I/O indicate that the overhead introduced by the IOTLB misses is negligible.}
\label{tab:netperf}
\end{table}

In fact, it is suggested that the overhead introduced by the IOTLB misses can be measured by using a high-speed I/O device (i.e., Intel's I/O Acceleration Technology~\cite{lauritzenintel}) so as to do DMA copy operations in a pseudo pass-through mode of IOMMU~\cite{amit2012iommu}. The experiment results indicate that DMA-copy speed is largely reduced by the IOTLB misses.
Moshe Malka et al. use rIOMMU with ibverbs library~\cite{ibverbsevaluation,kerr2011dissecting} to establish a high-performance setting in order to measure the cost of one IOTLB miss.

We believe that the \name can effectively reduce the overhead in the above high-speed settings, and we also plan to conduct experiments in the future.
%However, currently we do not have a machine with the Intel I/O acceleration technology. Thus, we have to postpone this experiment. 

%play an important role in I/O environments requiring high performance, which will be evaluated in our future work by utilizing the previous approaches and the I/O high-speed device.
%Both studies aim to create high performance settings that reduce the DMA transactions to the magnitude of $\upmu$s so that IOTLB becomes the dominant factor.


%\begin{figure}[htp]
%\centering
%\includegraphics[scale=0.55]{image/iotlbflush.png} \\
%\includegraphics[width=0.5\textwidth]{image/macro/netperf.png} \\
%\caption{Netperf}
%\label{fig:netperf}
%\end{figure}
%|p{1.7cm}|p{1.8cm}|p{1.7cm}


%\begin{figure*}[!t]
%\centering
%\subfigure[PGD Alloc]{
%\label{fig:subfig:a}
%\includegraphics[width=0.5\textwidth]{image/micro/PGDalloc.png}}
%\hspace{1in}
%\subfigure[PGD Free]{
%\label{fig:subfig:b}
%\includegraphics[width=0.5\textwidth]{image/micro/PGDfree.png}}
%\caption{Both page-table cache groups costs much less CPU cycles}
%\label{fig:PGDtime} %% label for entire figure
%\end{figure*}

%\begin{figure}
%\centering
%\subfigure[PGD]{
%\label{fig:subfig:a}
%\includegraphics[width=0.5\textwidth]{image/micro/PGDtime.png}}
%\hspace{1in}
%\subfigure[PMD]{
%\label{fig:subfig:b}
%\includegraphics[width=0.5\textwidth]{image/micro/PMDtime.png}}
%\hspace{1in}
%\subfigure[PTE]{
%\label{fig:subfig:c}
%\includegraphics[width=0.5\textwidth]{image/micro/PTEtime.png}}
%\caption{CPU Usage for Each Level of Page Table}
%\label{fig:PGtime} %% label for entire figure
%\end{figure}

%\begin{figure}
%\centering
%\subfigure[PGD]{
%\label{fig:subfig:a}
%\includegraphics[width=0.5\textwidth]{image/micro/dyn_PGDpool.png}}
%\hspace{1in}
%\subfigure[PMD]{
%\label{fig:subfig:b}
%\includegraphics[width=0.5\textwidth]{image/micro/dyn_PMDpool.png}}
%\hspace{1in}
%\subfigure[PTE]{
%\label{fig:subfig:c}
%\includegraphics[width=0.5\textwidth]{image/micro/dyn_PTEpool.png}}
%\caption{Cache Pools Size for Cache-Dynamic-Enabled Group}
%\label{fig:dynPGpool} %% label for entire figure
%\end{figure}

%\begin{figure}
%\centering
%\subfigure[PGD]{
%\label{fig:subfig:a}
%\includegraphics[width=0.5\textwidth]{image/micro/pre_PGDpool.png}}
%\hspace{1in}
%\subfigure[PMD]{
%\label{fig:subfig:b}
%\includegraphics[width=0.5\textwidth]{image/micro/pre_PMDpool.png}}
%\hspace{1in}
%\subfigure[PTE]{
%\label{fig:subfig:c}
%\includegraphics[width=0.5\textwidth]{image/micro/pre_PTEpool.png}}
%\caption{Cache Pools Size for Cache-Pre-Enabled Group}
%\label{fig:prePGpool} %% label for entire figure
%\end{figure}

%PGD & $5$  & $4$   & $>1:1$ \\ \hline
%PMD & $26$ & $12$   & $>2:1$ \\ \hline
%PTE & $145$ & $177$ & $<1:1$ \\ \hline
%Total & $176$ & $193$ & $<1:1$ \\ \hline

%PGD & $4$  & $2$   & $2:1$ \\ \hline
%PMD & $20$ & $4$   & $5:1$ \\ \hline
%PTE & $136$ & $182$ & $<1:1$ \\ \hline
%Total & $160$ & $188$ & $<1:1$ \\ \hline
%\begin{table}[!ht]
%\footnotesize
%\begin{center}
%\begin{tabular}{|l|l|l|}
%\hline
%{\textbf{Levels of Page Table}} & {\textbf{Semi-writable Page (\#)}} & {\textbf{Page-Table Page (\#)}} \\ \hline
%L1 & $5$  & $4$ \\ \hline
%L2 & $26$ & $12$ \\ \hline
%L3 & $145$ & $177$ \\ \hline
%Total & $176$ & $193$ \\ \hline
%\end{tabular}
%\end{center}
%\caption{The memory usage of the \cache is small in the \prename group, only occupying 176 pages. }
%And the memory size of cached pages only takes up to 0.2\% of the tool's memory.
%\label{tab:prePGpool}
%\end{table}

%\begin{table}[!ht]
%\footnotesize
%\begin{center}
%\begin{tabular}{|l|l|l|}
%\hline
%{\textbf{Levels of Page Table}} & {\textbf{Semi-writable Page (\#)}} & {\textbf{Page-Table Page (\#)}} \\ \hline
%L1 & $4$  & $2$ \\ \hline
%L2 & $20$ & $4$  \\ \hline
%L3 & $136$ & $182$ \\ \hline
%Total & $160$ & $188$ \\ \hline
%\end{tabular}
%\end{center}
%\caption{}
%\label{tab:dynPGpool}
%\end{table}

\section{\name Discussion} \label{sec:dis}

As talked before, when to free pages in the cache pool relies on both the absolute memory size of the cache and the proportion between the cache size and page-table size. Default thresholds of the two factors are determined by a specific setting, where both thresholds are measured in the busy setting (created by a specific stress tool) when no pages are freed to the buddy system. Therefore, measurements of the thresholds are heavily dependent on a specific setting, which may not work in other situations. If the thresholds are not set appropriately, freeing pages will occur often, thus causing IOTLB flush. This is why we provide an interface for users to manually modify the default values of both thresholds. Nevertheless, the approach of developing the interface is not flexible enough to control the cache size and we plan to put forward a self-adaption algorithm in the future work. Basically, the algorithm is invoked periodically, adjusting the memory usage of the cache pool according to both the memory page number of created page tables of a target application and the frequency of IOTLB-flush. When users dynamically enable the cache mechanism for the application, the algorithm could automatically initialise the absolute memory usage by scanning the page numbers in cache and in page tables and also determine the proportion between them. Note that the proportion differs in each level of cache, since PTE is used more often than both PGD and PMD. If IOTLB is found out to be flushing during the running period, the algorithm will appropriately increase the threshold of the total memory size. If the cache is beyond a specific percentage of the application's whole memory size, the algorithm is supposed to free the exceeded memory pages even if the frequency of IOTLB flush does not drop to zero.

%has the properties as follows.
%\begin{enumerate}
%\item (P1) Frequency of IOTLB flush will reach the zero level as soon as possible.
%\item (P2) Memory usage of the cache pool is under control.
%\item (P3)
%\item (P4)
%\end{enumerate}

%\zhi{Macrobenchmark results of netperf do not reveal IOTLB's impacts on the DMA transactions.}

%\section{Related Work} \label{sec:rel}
%introduce full virtualization, then move onto PV for performance and finally introduce the security and performance issues.
VMware~\cite{devine2002virtualization} implements a full virtualization of the underlying computer hardware and allows unmodified guest OSes to execute on a hypervisor, which has degraded the system performance. Denali~\cite{whitaker2002scale} is the first to develop paravirtualization techniques to achieve high performance for modified VMs running network services while Xen~\cite{barham2003xen} is intended to support real operating systems hosting industry standard applications, which makes Xen become popular and widely used in cloud computing. And how to improve its security and performance becomes a major concern.

%talk about how to improve Xen's security.
Murray \emph{et al.}~\cite{disaggregation} manage to reduce trusted computing base (TCB) of a Xen-based system, which moves the VM-building component from the privileged VM, namely \emph{Domain 0} into a small and trusted compartment.
Xoar~\cite{colp2011breaking} is proposed to protect Xen hypervisor by breaking the \emph{Domain 0} into several single-purpose components, and each component is configured to expose its dedicated interfaces to VMs and have the least required-privilege access to the hypervisor, also resulting in a reduction of TCB. CloudVisor~\cite{zhang2011cloudvisor} is introduced to prevent leakage of users's data inside a VM by breaking Xen hypervisor into both a resource management module and a nested security monitor and the monitor is responsible for providing protection to the VMs, largely improving Xen's security. 

%improve I/O performance for paravirtual I/O method
As I/O activity is an important performance factor in virtualized environments, the paravirtual I/O method introduced by Xen is efficient to transfer I/O data. Instead of emulating hardware devices, Xen asks the \emph{backend} of a device running in the \emph{driver domain} to communicate with the \emph{frontend} of that device residing in a guest domain by passing the data info through shared-memory, etc. Har'El \emph{et al.}~\cite{har2013efficient} claim to provide a more efficient paravirtual I/O system by combing a fine-grained I/O scheduling and exitless notifications with separate cores, each core dedicated to handling one domain's I/O requests. Besides, to approach bare-metal performance for VMs that interact with I/O devices directly, ELI~\cite{eli} is presented to remove the hypervisor from the I/O interrupt handling path while handle the interrupts within VMs securely.

%talk about IOMMU performance when it is armed by Xen
As the paravirtual I/O method is not secure enough for DMA access~\cite{disaggregation}, IOMMU (AMD-Vi~\cite{amdvt} or Intel VT-d~\cite{intelvt}) is armed by Xen to prevent buggy device drivers from overwriting system's memory, which subsequently introduces new I/O performance issues. On top of that, Willmann \emph{et al.}~\cite{willmann2008protection} proposes new strategies for Xen to configure IOMMU in order to reduce I/O performance overhead without sacrificing Xen's security. Particularly, Amit \emph{et al.}~\cite{amit2012iommu} and Malka \emph{et al.}~\cite{malka2015riommu} deeply analyze the role of IOMMU's IOTLB in DMA operations and quantifies bottleneck overhead of IOTLB in the high I/O performance environments.

%introduce our work
In our work, we are focusing on the page table (de)allocations of paravirtualized OS. When an OS is ported to Xen, there exists long execution paths of the guest page table (de)allocations and additional IOTLB flushes due to the security validations for page table (de)allocations. Because of the two performance issues, \name is presented to efficiently cut down the execution length and completely eliminate IOTLB flushes, resulting in better performance for both OS and IOMMU.

%Although the hypervisor provides valuable services for memory access from both sides, it is far from enough.
%so as to reduce the performance degradation for the OS inside a VM while prevent illicit access or faults from the OS, achieving a good tradeoff between performance and safety for software access.
%NoHype~\cite{keller2010nohype}
%in its original design, Xen uses an efficient
%Ben-Yehuda~\cite{ben2008xen} talks about the I/O virtualization of Xen by IOMMU~\cite{intelvt,amdvt}, which not only allows direct access to I/O devices by untrusted VMs but prevents buggy device drivers from overwriting system's memory, thereby largely improving the system's availability and reliability for DMA access. 
%\section{Conclusion} \label{sec:con}

Return-Oriented Programming (ROP) is a powerful attack that is able to perform Turing-complete computation without injecting any code. It enables the adversary to perform malicious operations that is not intended by the original programmer.
Existing approaches either need the side information (e.g., source code or debugging information), break the binary integrity, or suffer high performance overhead.
In this paper, we have designed and implemented a novel system, named as \name, which provide a generic solution to prevent all kinds of ROP attacks. \name requires no source code, no compiler support and no binary rewriting. The security and performance experiment results show that \name is able to detect any forms of ROP attacks with 2\% performance overhead on average. In addition, we observe no false positive or false negative in all our experiments.


%ACKNOWLEDGMENTS are optional
%\section{Acknowledgments}




%Now we get serious and fill in those references.  Remember you will
%have to run latex twice on the document in order to resolve those
%cite tags you met earlier.  This is where they get resolved.
%We've preserved some real ones in addition to the template-speak.
%After the bibliography you are DONE.

%{\footnotesize \bibliographystyle{acm}
%\bibliography{../common/bibliography}}
{
%\footnotesize
\bibliographystyle{plain}
\bibliography{fastio}
}


\end{document}

IOMMUs have been pervasively deployed on paravirtualized systems for the protection of the hypervisor and the security-critical data structures, e.g., the shared page tables. According to our observations, certain updates of the guest VM's page tables that are supposed to be orthogonal to the device I/O performance, would surprisingly lead to a large number of IOTLB misses. It implies that the I/O performance of all peripheral devices will be affected by the seemingly unrelated guest page table updates. Unfortunately, no existing work uncovers this dependency and adjusts the design of the paravirtualized hypervisor and the guest operating system.







