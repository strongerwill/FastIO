\section{Related Work} \label{sec:rel}
%introduce full virtualization, then move onto PV for performance and finally introduce the security and performance issues.
VMware~\cite{devine2002virtualization} implements a full virtualization of the underlying computer hardware and allows unmodified guest OSes to execute on a hypervisor, which has degraded the system performance. Denali~\cite{whitaker2002scale} is the first to develop paravirtualization techniques to achieve high performance for modified VMs running network services while Xen~\cite{barham2003xen} is intended to support real operating systems hosting industry standard applications, which makes Xen become popular and widely used in cloud computing. And how to improve its security and performance becomes a major concern.

%talk about how to improve Xen's security.
Murray \emph{et al.}~\cite{disaggregation} manage to reduce trusted computing base (TCB) of a Xen-based system, which moves the VM-building component from the privileged VM, namely \emph{Domain 0} into a small and trusted compartment.
Xoar~\cite{colp2011breaking} is proposed to protect Xen hypervisor by breaking the \emph{Domain 0} into several single-purpose components, and each component is configured to expose its dedicated interfaces to VMs and have the least required-privilege access to the hypervisor, also resulting in a reduction of TCB. CloudVisor~\cite{zhang2011cloudvisor} is introduced to prevent leakage of users's data inside a VM by breaking Xen hypervisor into both a resource management module and a nested security monitor and the monitor is responsible for providing protection to the VMs, largely improving Xen's security. 

%improve I/O performance for paravirtual I/O method
As I/O activity is an important performance factor in virtualized environments, the paravirtual I/O method introduced by Xen is efficient to transfer I/O data. Instead of emulating hardware devices, Xen asks the \emph{backend} of a device running in the \emph{driver domain} to communicate with the \emph{frontend} of that device residing in a guest domain by passing the data info through shared-memory, etc. Har'El \emph{et al.}~\cite{har2013efficient} claim to provide a more efficient paravirtual I/O system by combing a fine-grained I/O scheduling and exitless notifications with separate cores, each core dedicated to handling one domain's I/O requests. Besides, to approach bare-metal performance for VMs that interact with I/O devices directly, ELI~\cite{eli} is presented to remove the hypervisor from the I/O interrupt handling path while handle the interrupts within VMs securely.

%talk about IOMMU performance when it is armed by Xen
As the paravirtual I/O method is not secure enough for DMA access~\cite{disaggregation}, IOMMU (AMD-Vi~\cite{amdvt} or Intel VT-d~\cite{intelvt}) is armed by Xen to prevent buggy device drivers from overwriting system's memory, which subsequently introduces new I/O performance issues. On top of that, Willmann \emph{et al.}~\cite{willmann2008protection} proposes new strategies for Xen to configure IOMMU in order to reduce I/O performance overhead without sacrificing Xen's security. Particularly, Amit \emph{et al.}~\cite{amit2012iommu} and Malka \emph{et al.}~\cite{malka2015riommu} deeply analyze the role of IOMMU's IOTLB in DMA operations and quantifies bottleneck overhead of IOTLB in the high I/O performance environments.

%introduce our work
In our work, we are focusing on the page table (de)allocations of paravirtualized OS. When an OS is ported to Xen, there exists long execution paths of the guest page table (de)allocations and additional IOTLB flushes due to the security validations for page table (de)allocations. Because of the two performance issues, \name is presented to efficiently cut down the execution length and completely eliminate IOTLB flushes, resulting in better performance for both OS and IOMMU.

%Although the hypervisor provides valuable services for memory access from both sides, it is far from enough.
%so as to reduce the performance degradation for the OS inside a VM while prevent illicit access or faults from the OS, achieving a good tradeoff between performance and safety for software access.
%NoHype~\cite{keller2010nohype}
%in its original design, Xen uses an efficient
%Ben-Yehuda~\cite{ben2008xen} talks about the I/O virtualization of Xen by IOMMU~\cite{intelvt,amdvt}, which not only allows direct access to I/O devices by untrusted VMs but prevents buggy device drivers from overwriting system's memory, thereby largely improving the system's availability and reliability for DMA access. 