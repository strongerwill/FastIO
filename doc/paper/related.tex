\section{Related Work} \label{sec:rel}
VMware~\cite{devine2002virtualization} implements a full virtualization of the underlying computer hardware and allows unmodified guest OSes to execute on a hypervisor. However, OS executing inside a VM has degraded its system performance. Denali~\cite{whitaker2002scale} is the first to use paravirtualization techniques to achieve high performance for modified virtual machines running network services while Xen~\cite{barham2003xen} is intended to support real operating systems hosting industry standard applications, which makes Xen become popular and widely used in cloud computing. As a result, how to improve its security becomes a major concern.

Murray \emph{et al.}~\cite{disaggregation} manage to reduce trusted computing base (TCB) of a Xen-based system, which moves the VM-building component from the privileged VM, namely \emph{Dom0} into a small and trusted compartment.
Xoar~\cite{colp2011breaking} is proposed to protect Xen hypervisor by breaking a monolithic control VM into several single-purpose components, and each component is configured to expose fewer interfaces to VMs and have the least privileged restricted access to the hypervisor. CloudVisor~\cite{zhang2011cloudvisor} is introduced to prevent leakage of users's data inside a VM by breaking Xen hypervisor into both a resource management module and a nested security monitor and the monitor is responsible for providing protection to the VMs.
%NoHype~\cite{keller2010nohype}
%in its original design, Xen uses an efficient 
Besides, the paravirtual I/O method introduced by Xen is efficient to transfer I/O data between the \emph{backend} of a device running in the \emph{driver domain} and the \emph{frontend} of that device residing in a guest domain. Har'El \emph{et al.}~\cite{har2013efficient} claim to provide a more efficient paravirtual I/O system by combing a fine- grained I/O scheduling and exitless notifications with separate cores, each core dedicated to handling one domain's I/O requests. To approach bare-metal performance for each guest virtual machine, ELI~\cite{eli} is proposed to remove the host from the I/O interrupt handling path.

Since the paravirtual I/O method is not secure enough for DMA access, IOMMU~\cite{intelvt,amdvt} is armed by Xen to defend against DMA attacks, which subsequently introduces new I/O performance issues. Willmann \emph{et al.}~\cite{willmann2008protection} introduces new strategies for Xen to configure IOMMU in order to reduce I/O performance overhead without sacrificing Xen's security. Amit \emph{et al.}~\cite{amit2012iommu} and Malka \emph{et al.}~\cite{malka2015riommu} deeply analyze the role of IOTLB in DMA operations and quantifies its bottleneck overhead in the high I/O performance environments.

%Ben-Yehuda~\cite{ben2008xen} talks about the I/O virtualization of Xen by IOMMU~\cite{intelvt,amdvt}, which not only allows direct access to I/O devices by untrusted VMs but prevents buggy device drivers from overwriting system's memory, thereby largely improving the system's availability and reliability for DMA access.

In our work, we focus on Xen's original design about page table management. When an OS is ported to Xen, the OS is supposed to take responsibility for (de)allocating its own page tables while Xen is only involved in the page table updates so as to reduce the performance degradation for the OS inside a VM while prevent illicit access or faults from the OS, achieving a good tradeoff between performance and safety for software access.

%Although the hypervisor provides valuable services for memory access from both sides, it is far from enough.


